\newpage
\section{Career reflection}

\subsection{Description}
\begin{itemize}
    \item What meaningful tasks, projects or initiatives were you involved in? Why do you see them as meaningful?  What problems did you solve or make progress on?
    \item Where did you feel most challenged (stretched) during this internship? Why was this challenging and how did you overcome this?
    \item What positive feedback and/or constructive criticism did you receive from your colleague(s), supervisor(s)? How did you respond to them? How did you implement the feedback?
\end{itemize}

\subsection{Content}

\subsubsection{The tasks I involved}

\begin{itemize}
    \item 
The first component I was working on was the Save feature which is responsible for saving the file that user just edited.
I improved the feature by adding new command "Save without Formatting" to the framework, allowing user to save their changes without applying any formatters.
After implementing the command, I found other issues in the same module so I proposed by creating issues to track the problems, and fixed them.
    \item
Next, I moved to working with "Search-In-Workspace" (SIW) module of the framework which the main functionality is to allow users to search for a term, then displays the results out for review.
Here, I made the improvement by adding the ability for the framework to also search unsaved changes from users. Furthermore, I was proposing to the team and asked to extend the functionality of this component even more.
In the end, I believe the features I added helps a lot to improve user experience since they now have a search component that is much more dynamic than before.
    \item 
Recently, the project is making a migration from Travis CI to GitHub Actions for CI-CD (services for continuous integration and continuous delivery).
I proposed to be a part of this effort. As a result, I was working to re-create the pipelines for several projects in Theia:
        \begin{itemize}
            \item https://github.com/theia-ide/theia-apps
            \item https://github.com/eclipse-theia/theia-cpp-extensions
            \item https://github.com/theia-ide/generator-theia-extension
        \end{itemize}
It was my first time using CI-CD tool like GitHub, but I picked up on the way when I was implementing the pipelines.
\end{itemize}

\newpage
\subsubsection{Challenges}

The challenge I encountered is the complexity level of the code base. I never worked in such a huge code base / project like this before. Also, the project is evolving everyday with active contributions coming in daily.
This made it difficult to navigate at first. However, Ericsson has a buddy system, where new intern gets assigned with an onboarding buddy who will support the ramp-up process.
In my case, he was super helpful in guiding me and helping me to understand the project better.
In addition, I started the internship by fixing several small issues. Those first steps inspired me to take on more complicated tasks later on.

\subsubsection{Respond to feedback}
My colleagues helped me a lots by criticizing my code in a constructive manner. Through that, I have improved myself to write cleaner code and to express my idea by code more efficient and cleaer.
That also helped me to learn to review other people code detailedly. This serves as a concrete foundation that helped me to gain confident to review code from other people later in the internship.


